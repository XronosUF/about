\documentclass{ximera}
\title{About Ximera}

\begin{document}
\begin{abstract}
  Ximera is a free and open-source platform for creating and sharing
  interactive online course materials.
\end{abstract}
\maketitle


The Ximera Project (pronounced
\href{http://en.wikipedia.org/wiki/Chimera_(mythology)}{chimera}, with a
Greek \href{http://en.wikipedia.org/wiki/Chi_(letter)}{chi}) is a
\href{https://github.com/kisonecat/ximera}{free and open-source
platform} for creating and sharing interactive online course materials.
The goal of Ximera is to make it easier for authors familiar with
\href{http://en.wikipedia.org/wiki/LaTeX}{LaTeX} to create interactive
online content and to provide educators and researchers with
quantitative data on student performance and involvement.

Key features of our platform include

\begin{itemize}
\item
  \textbf{A content conversion tool and LaTeX document class.} Ximera
  gives authors the ability to create a \emph{single} LaTeX document
  that can be compiled as both a
  \href{http://en.wikipedia.org/wiki/Portable_Document_Format}{PDF} and
  an \href{http://en.wikipedia.org/wiki/HTML_file}{HTML} file. This
  allows authors to create course materials that can be used as handouts
  or as interactive webpages.
\item
  \textbf{Web hosting and content repository.} All online content
  generated using Ximera will be hosted for free. In doing so, the goal
  is to encourage the creation of a large community of active authors
  that can generate and share course materials without the worry of more
  technical issues.
\item
  \textbf{A JavaScript library.} Having created quite a few interactive
  elements already with
  \href{http://en.wikipedia.org/wiki/JavaScript}{JavaScript}, Ximera
  also acts as a library providing common functionality that can be used
  to produce virtual manipulatives. Authors can contribute to the
  library or borrow from it to enhance their own course materials.
\item
  \textbf{Modular content.} Ximera gives educators the ability to pick
  and choose from the course materials available online, allowing for
  the construction of custom courses.
\item
  \textbf{Feedback through data analysis.} Ximera does more than just
  act as a hosting site for authors and educators, it also has a
  powerful backend that collects data and analyses how students interact
  with the content. Educators and researchers can receive feedback from
  students on a variety of different statistics, from basic mean
  computations on a set of questions to the amount of time students
  spend interacting with a particular manipulative.
\end{itemize}

\subsubsection{Support Ximera}\label{support-ximera}

Building the backend technology, providing the servers, and handling the
technical issues for our educators all require the work of many
people---all that ultimately means that Ximera depends on you. If you
are interested in partnering with Ximera to provide education technology
to the world, please \href{/about/contact}{contact us}.

To get involved right away, there are several simple things you can do.

\begin{itemize}
\item
  Download our LaTeX tools from
  \href{https://github.com/bartsnapp/ximeraLatex}{GitHub} and use them
  to write a LaTeX document and try our conversion process.
\item
  Write JavaScript interactive elements and share them with others.
\item
  Download our source code from
  \href{https://github.com/kisonecat/ximera}{GitHub} and contribute to
  the content conversion tool.
\end{itemize}

\end{document}
